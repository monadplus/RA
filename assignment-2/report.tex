\documentclass[12pt, a4paper]{article} % book, report, article, letter, slides
                                       % letterpaper/a4paper, 10pt/11pt/12pt, twocolumn/twoside/landscape/draft

%%%%%%%%%%%%%%%% PACKAGES %%%%%%%%%%%%%%%%%%%%%

\usepackage[utf8]{inputenc} % encoding

\usepackage[english]{babel} % use special characters and also translates some elements within the document.

\usepackage{amsmath}        % Math
\usepackage{amsthm}         % Math, \newtheorem, \proof, etc
\usepackage{amssymb}        % Math, extended collection
\usepackage{bm}             % $\bm{D + C}$
\newtheorem{theorem}{Theorem}[section]     % \begin{theorem}\label{t:label}  \end{theorem}<Paste>
\newtheorem{corollary}{Corollary}[theorem]
\newtheorem{lemma}[theorem]{Lemma}
\newenvironment{claim}[1]{\par\noindent\underline{Claim:}\space#1}{}
\newenvironment{claimproof}[1]{\par\noindent\underline{Proof:}\space#1}{\hfill $\blacksquare$}
\newcommand\expect[1]{\mathbb{E}[#1]}

\usepackage{hyperref}       % Hyperlinks \url{url} or \href{url}{name}

\usepackage{parskip}        % \par starts on left (not idented)

\usepackage{abstract}       % Abstract

\usepackage{enumitem}       % \item[$ast$] Foo

\usepackage{tocbibind}      % Adds the bibliography to the table of contents (automatically)

\usepackage{graphicx}       % Images
\graphicspath{{./images/}}

\usepackage[vlined,ruled]{algorithm2e} % pseudo-code

% \usepackage[document]{ragged2e}  % Left-aligned (whole document)
% \begin{...} ... \end{...}   flushleft, flushright, center

%%%%%%%%%%%%%%%% CODE %%%%%%%%%%%%%%%%%%%%%

\usepackage{minted}         % Code listing
% \mint{html}|<h2>Something <b>here</b></h2>|
% \inputminted{octave}{BitXorMatrix.m}

%\begin{listing}[H]
  %\begin{minted}[xleftmargin=20pt,linenos,bgcolor=codegray]{haskell}
  %\end{minted}
  %\caption{Example of a listing.}
  %\label{lst:example} % You can reference it by \ref{lst:example}
%\end{listing}

\newcommand{\code}[1]{\texttt{#1}} % Define \code{foo.hs} environment

%%%%%%%%%%%%%%%% COLOURS %%%%%%%%%%%%%%%%%%%%%

\usepackage{xcolor}         % Colours \definecolor, \color{codegray}
\definecolor{codegray}{rgb}{0.9, 0.9, 0.9}
% \color{codegray} ... ...
% \textcolor{red}{easily}

%%%%%%%%%%%%%%%% CONFIG %%%%%%%%%%%%%%%%%%%%%

\renewcommand{\absnamepos}{flushleft}
\setlength{\absleftindent}{0pt}
\setlength{\absrightindent}{0pt}

%%%%%%%%%%%%%%%% GLOSSARIES %%%%%%%%%%%%%%%%%%%%%

%\usepackage{glossaries}

%\makeglossaries % before entries

%\newglossaryentry{latex}{
    %name=latex,
    %description={Is a mark up language specially suited
    %for scientific documents}
%}

% Referene to a glossary \gls{latex}
% Print glossaries \printglossaries

\usepackage[acronym]{glossaries} %

% \acrshort{name}
% \acrfull{name}
\newacronym{kcol}{$k$-COL}{$k$-coloring problem}
\newacronym{scol}{SEARCH-$k$-COL}{Search $k$-coloring problem}
\newacronym{2col}{$2$-COL}{$2$-coloring problem}
\newacronym{e2sat}{$E2$-SAT}{Exactly 2-SAT}

%%%%%%%%%%%%%%%%%%%%%

\usepackage[displaymath,textmath,sections,graphics,floats]{preview}
% \PreviewEnvironment{enumerate}
\PreviewEnvironment{tabular}

%%%%%%%%%%%%%%%% HEADER %%%%%%%%%%%%%%%%%%%%%

\usepackage{fancyhdr}
\pagestyle{fancy}
\fancyhf{}
\rhead{Arnau Abella Gassol}
\lhead{Randomized Algorithms}
\rfoot{Page \thepage}

%%%%%%%%%%%%%%%% TITLE %%%%%%%%%%%%%%%%%%%%%

\title{%
  Randomized Algorithms\\
  \large{Second Assignment}
}
\author{%
  Arnau Abella \\
  \large{Universitat Polit\`ecnica de Catalunya}
}
\date{\today}

%%%%%%%%%%%%%%%% DOCUMENT %%%%%%%%%%%%%%%%%%%%%

\begin{document}

\maketitle

%%%%%%% Exercise 1

\section*{Exercise 1}%
\label{sec:exercise_1}

The problem can be generalized to the \textit{bins and balls} problems where the bins are the squared boxed and the balls are the points in the unit space.

Let $n$ be the number of points, $m = n/\log^{2}n$ be the number of square boxes and $X_{i}$ be the random variable counting the number of balls into bin $i$ where $X_{i} \sim B(n, \frac{1}{m})$.

In order to prove that for any $\epsilon \in (0,1)$, \textit{w.h.p} a collection of $n$ points taken \textit{u.a.r} in the unit square is $\epsilon$-nice we need to show that the number of balls in each bin is concentrated in the range $[(1-\epsilon) \log^{2}n, (1 + \epsilon)\log^{2}n]$.

Recall \textit{Chebyshev's Inequality}: For any $a > 0$,

\begin{equation*}
  Pr(|X - \expect{X} | \geq a) \leq \frac{Var[X]}{a^2}
\end{equation*}

The average load in a bin is $\mu = \expect{X_{i}} = n/m = \log^{2} n$. Then, applying \textit{Chebyshev's Inequality} we get

\begin{align*}
  \Pr(|X - \log^{2}n | \geq \epsilon \log^{2}n) &\leq \frac{\log^{2}n \cdot (1 - \log^{2}n/n)}{(\epsilon\log^{2}n)^2} \\
  &= \frac{1 - (\log^{2}n/n)}{(\epsilon^2\log^{2}n)}
\end{align*}

For any $\epsilon$, when $n \to \infty$, the probability that a bin has less than ${(1-\epsilon) \log^{2}n}$ balls or more than $(1 + \epsilon)\log^{2}n$ balls tends to $0$

\begin{equation*}
  \lim_{n \to \infty} \frac{1 - (\log^{2}n/n)}{(\epsilon^2\log^{2}n)} =
  \lim_{n \to \infty} \frac{1}{n} = 0
\end{equation*}

Then, \textit{w.h.p} the number of balls in each bin is in the interval ${[(1-\epsilon) \log^{2}n, (1 + \epsilon)\log^{2}n]}$ and so, a collection of $n$ points taken \textit{u.a.r} in the unit square is $\epsilon$-nice.

%%%%%%% Exercise 2

\section*{Exercise 2}%
\label{sec:exercise_2}

%%%%%%% Exercise 3

\section*{Exercise 3}%
\label{sec:exercise_3}

%%%%%%%%%%%%%%%% BIBLIOGRAPHY %%%%%%%%%%%%%%%%%%%%%

% \bibliographystyle{unsrt} % abbrv, aplha, plain, abstract, apa, unsrt,
% \bibliography{refs}

\end{document}
