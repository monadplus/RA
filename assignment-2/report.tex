\documentclass[12pt, a4paper]{article} % book, report, article, letter, slides
                                       % letterpaper/a4paper, 10pt/11pt/12pt, twocolumn/twoside/landscape/draft

%%%%%%%%%%%%%%%% PACKAGES %%%%%%%%%%%%%%%%%%%%%

\usepackage[utf8]{inputenc} % encoding

\usepackage[english]{babel} % use special characters and also translates some elements within the document.

\usepackage{amsmath}        % Math
\usepackage{amsthm}         % Math, \newtheorem, \proof, etc
\usepackage{amssymb}        % Math, extended collection
\usepackage{bm}             % $\bm{D + C}$
\newtheorem{theorem}{Theorem}[section]     % \begin{theorem}\label{t:label}  \end{theorem}<Paste>
\newtheorem{corollary}{Corollary}[theorem]
\newtheorem{lemma}[theorem]{Lemma}
\newenvironment{claim}[1]{\par\noindent\underline{Claim:}\space#1}{}
\newenvironment{claimproof}[1]{\par\noindent\underline{Proof:}\space#1}{\hfill $\blacksquare$}
\newcommand\expect[1]{\mathbb{E}[#1]}
\newcommand\angles[1]{\langle #1 \rangle}

\usepackage{hyperref}       % Hyperlinks \url{url} or \href{url}{name}

\usepackage{parskip}        % \par starts on left (not idented)

\usepackage{abstract}       % Abstract

\usepackage{enumitem}       % \item[$ast$] Foo

\usepackage{tocbibind}      % Adds the bibliography to the table of contents (automatically)

\usepackage{graphicx}       % Images
\graphicspath{{./images/}}

\usepackage[vlined,ruled]{algorithm2e} % pseudo-code

% \usepackage[document]{ragged2e}  % Left-aligned (whole document)
% \begin{...} ... \end{...}   flushleft, flushright, center

%%%%%%%%%%%%%%%% CODE %%%%%%%%%%%%%%%%%%%%%

\usepackage{minted}         % Code listing
% \mint{html}|<h2>Something <b>here</b></h2>|
% \inputminted{octave}{BitXorMatrix.m}

%\begin{listing}[H]
  %\begin{minted}[xleftmargin=20pt,linenos,bgcolor=codegray]{haskell}
  %\end{minted}
  %\caption{Example of a listing.}
  %\label{lst:example} % You can reference it by \ref{lst:example}
%\end{listing}

\newcommand{\code}[1]{\texttt{#1}} % Define \code{foo.hs} environment

%%%%%%%%%%%%%%%% COLOURS %%%%%%%%%%%%%%%%%%%%%

\usepackage{xcolor}         % Colours \definecolor, \color{codegray}
\definecolor{codegray}{rgb}{0.9, 0.9, 0.9}
% \color{codegray} ... ...
% \textcolor{red}{easily}

%%%%%%%%%%%%%%%% CONFIG %%%%%%%%%%%%%%%%%%%%%

\renewcommand{\absnamepos}{flushleft}
\setlength{\absleftindent}{0pt}
\setlength{\absrightindent}{0pt}

%%%%%%%%%%%%%%%% GLOSSARIES %%%%%%%%%%%%%%%%%%%%%

%\usepackage{glossaries}

%\makeglossaries % before entries

%\newglossaryentry{latex}{
    %name=latex,
    %description={Is a mark up language specially suited
    %for scientific documents}
%}

% Referene to a glossary \gls{latex}
% Print glossaries \printglossaries

\usepackage[acronym]{glossaries} %

% \acrshort{name}
% \acrfull{name}
\newacronym{kcol}{$k$-COL}{$k$-coloring problem}
\newacronym{scol}{SEARCH-$k$-COL}{Search $k$-coloring problem}
\newacronym{2col}{$2$-COL}{$2$-coloring problem}
\newacronym{e2sat}{$E2$-SAT}{Exactly 2-SAT}

%%%%%%%%%%%%%%%%%%%%%

\usepackage[displaymath,textmath,sections,graphics,floats]{preview}
% \PreviewEnvironment{enumerate}
\PreviewEnvironment{tabular}

%%%%%%%%%%%%%%%% HEADER %%%%%%%%%%%%%%%%%%%%%

\usepackage{fancyhdr}
\pagestyle{fancy}
\fancyhf{}
\rhead{Arnau Abella Gassol}
\lhead{Randomized Algorithms}
\rfoot{Page \thepage}

%%%%%%%%%%%%%%%% TITLE %%%%%%%%%%%%%%%%%%%%%

\title{%
  Randomized Algorithms\\
  \large{Second Assignment}
}
\author{%
  Arnau Abella \\
  \large{Universitat Polit\`ecnica de Catalunya}
}
\date{\today}

%%%%%%%%%%%%%%%% DOCUMENT %%%%%%%%%%%%%%%%%%%%%

\begin{document}

\maketitle

%%%%%%% Exercise 1

\section*{Exercise 1}%
\label{sec:exercise_1}

The problem can be generalized to the \textit{bins and balls} problems where the bins are the squared boxed and the balls are the points in the unit space.

Let $n$ be the number of points, $m = n/\log^{2}n$ be the number of square boxes and $X_{i}$ be the random variable counting the number of balls into bin $i$ where $X_{i} \sim B(n, \frac{1}{m})$.

In order to prove that for any $\epsilon \in (0,1)$, \textit{w.h.p} a collection of $n$ points taken \textit{u.a.r} in the unit square is $\epsilon$-nice we need to show that the number of balls in each bin is concentrated in the range $[(1-\epsilon) \log^{2}n, (1 + \epsilon)\log^{2}n]$.

Recall \textit{Chebyshev's Inequality}: For any $a > 0$,

\begin{equation*}
  Pr(|X - \expect{X} | \geq a) \leq \frac{Var[X]}{a^2}
\end{equation*}

The average load in a bin is $\mu = \expect{X_{i}} = n/m = \log^{2} n$. Then, applying \textit{Chebyshev's Inequality} we get

\begin{align*}
  \Pr(|X - \log^{2}n | \geq \epsilon \log^{2}n) &\leq \frac{\log^{2}n \cdot (1 - \log^{2}n/n)}{(\epsilon\log^{2}n)^2} \\
  &= \frac{1 - (\log^{2}n/n)}{(\epsilon^2\log^{2}n)}
\end{align*}

For any $\epsilon \in (0,1)$, when $n \to \infty$, the probability that a bin has less than ${(1-\epsilon) \log^{2}n}$ balls or more than $(1 + \epsilon)\log^{2}n$ balls tends to $0$

\begin{equation*}
  \lim_{n \to \infty} \frac{1 - (\log^{2}n/n)}{(\epsilon^2\log^{2}n)} =
  \lim_{n \to \infty} \frac{\epsilon^{-1}}{n} = 0
\end{equation*}

Then, \textit{w.h.p} the number of balls in each bin is in the interval ${[(1-\epsilon) \log^{2}n, (1 + \epsilon)\log^{2}n]}$ and so, a collection of $n$ points taken \textit{u.a.r} in the unit square is $\epsilon$-nice.

%%%%%%% Exercise 2

\section*{Exercise 2}%
\label{sec:exercise_2}

\begin{enumerate}[label=(\alph*)]
  \item Show using Chebyshev that $(r^{2}/\epsilon^{2}\delta)$ samples are sufficient to solve the problem.

  \textbf{Solution} Let $r = \sqrt{Var[X]}/\expect{X}$.

    We are going to bound the probability of \textbf{not} being in the \textit{confidence interval} using Chebyshev's Inequality,

    \begin{align*}
      Pr((\sum_{i=1}^{t}X_{i})/t \ge (1+\epsilon)\expect{X}) &=  Pr(\sum_{i=1}^{t}X_{i} \ge t(1+\epsilon)\expect{X}) \\
      &= Pr(\sum_{i=1}^{t}X_{i} \ge t\expect{X} + \epsilon t \expect{X}) \\
    \end{align*}

    Let $Y = \sum_{i=1}^{t}X_{i}$ be a random variable. Notice that $\expect{Y} = \expect{\sum_{i=1}^{t}X_{i}} = \sum_{i=1}^{t}\expect{X_{i}} = \sum_{i=1}^{t}\expect{X} = t \cdot \expect{X}$. The same applies to the \textit{variance}.
    Then, using the new random variable $Y$ and the previous assumptions, we can bound the error using Chebyshev's Inequality,


    \begin{align*}
      Pr(\sum_{i=1}^{t}X_{i} \ge t\expect{X} + \epsilon t \expect{X}) &= Pr (Y \ge \expect{Y} + \epsilon \expect{Y}) \\
      &= Pr (Y - \expect{Y} \ge \epsilon \expect{Y}) \\
      &\le Pr (|Y - \expect{Y}| \ge \epsilon \expect{Y}) \\
      &\le \frac{Var(Y)}{(\epsilon \expect{Y})^{2}} \\
      &= \frac{Var(\sum_{t=1}^{t}X_{i})}{(\epsilon t \expect{Y})^{2}} \\
      &= \frac{t Var (X)}{\epsilon^{2} t^{2} \expect{Y}^{2}} \\
      &= \frac{t}{\epsilon^{2} t^{2}} \frac{Var(x)}{\expect{X}^{2}} \\
      &= \frac{r^{2}}{\epsilon^{2} t}
    \end{align*}

    Then, the number of samples $t$ to bound the probability of the estimation being in the interval to be at least $1-\delta$ is

    \begin{align*}
      Pr((\sum_{i=1}^{t}X_{i})/t < (1+\epsilon)\expect{X}) &= 1 - Pr((\sum_{i=1}^{t}X_{i})/t \ge (1+\epsilon)\expect{X}) \\
      &= 1 - \frac{r^{2}}{\epsilon^{2} t} \\
      &\ge 1 - \delta
    \end{align*}

    when $t = O(r^{2}/\epsilon^{2} \delta)$ the probability of being in the interval is $\ge 1 - \delta$.

  %%%%%%%%%%%%%
  \item Suppose that we need only a weak estimate that is within $\epsilon\expect{X}$ of $\expect{X}$, with probability at least $3/4$. Argue that $O(r^{2}/\epsilon^{2})$ samples are enough for this weak estimate.

  \textbf{Solution} When $\delta = 1/4$, the probability of having a weak estimate within $\epsilon\expect{X}$ of $\expect{X}$ is $1 - \delta = 3/4$. From the previous exercise, in order to achieve a probability of at least $1 - \delta$, the number of samples $t$ must be $t = O(\frac{r^{2}}{\epsilon^{2} \delta})$, replacing $\delta = 1/4$, we have $t = O(\frac{r^{2}}{\epsilon^{2}})$.
  %%%%%%%%%%%%%
  \item Show that, by taking the median of $O(\lg(1/\delta))$ weak estimates, we can obtain an estimate within $\epsilon\expect{X}$ of $\expect{X}$ with probability at least $(1 - \delta)$. Conclude that we need only $r^{2}\lg(1/\delta)/\epsilon^{2}$ samples.

    \textbf{Solution} Let's define a new random variable $X_{i}$ such that $X_{i} = 1$ if the $i$th weak estimate is outside of the confidence interval $\epsilon \expect{X}$ and $X_{i} = 0$ otherwise.

    Notice, if the median is within the confidence interval, less than half of the weak estimates are outside the confidence interval. Hence, $X_{i} \sim B(n, 1/4)$.

    Let $X = \sum_{i=1}^{m} X_{i}$ be the random variable counting the number of weak estimates that fall outside the confidence interval. To obtain an estimation with an error of $O(\delta)$, we will bound $Pr(X \ge m/2) \le \delta$ applying Chernoff,

    \begin{equation*}
      Pr(X \ge (1 + \delta)\expect{X}) \le e^{-\expect{X} \delta ^{2} / 3}
    \end{equation*}

    Using $\expect{X} = m/4$ and $\delta = 1$,

    \begin{equation*}
      Pr(X \ge m/2) \le e^{-m/12}
    \end{equation*}

    Chernoff bound gives a bound on the error of $e^{-m/12}$ and we would like to bound it to $\delta$. So, when $m = 12 \ln (1/\delta) = O(\ln (1/\delta))$, the probability of error is at most $\delta$.
  %%%%%%%%%%%%%
\end{enumerate}

%%%%%%% Exercise 3

\section*{Exercise 3}%
\label{sec:exercise_3}

\begin{enumerate}[label=(\alph*)]
  \item Show that if $\mathcal{H}$ is 2-universal, the it is universal.

  \textbf{Solution} If $\mathcal{H}$ is 2-universal, for any fixed pair of distinct keys $\langle x, y\rangle$ and for any $h \in \mathcal{H}$ chosen at random, the sequence $\langle h(x), h(y) \rangle$ is equally likely to be any of the $m^{2}$ sequences with elements draw from $\{0,1, \ldots , m - 1\}$. From the $m^{2}$ distinct sequences, $m$ of them are of the form $\langle x, x \rangle$ and so, the number of collisions for any random $h \in \mathcal{H}$ is $m/m^{2} = 1/m$ and the total number of collisions is $1/m \cdot |\mathcal{H}|$. Therefore, by definition, $\mathcal{H}$ is \textit{universal}.

  %%%%%%%%%%%%%
  \item Suppose that $U$ is the set of $n$-tuples of values drawn from $ \mathbb{Z}_{p} = \{0,1,\ldots, p - 1\}$, where $p$ is prime. Consider an element $x = \langle x_{0}, x_{1}, \ldots, x_{n-1}\rangle \in U$. For any $n$-tuple $a = \langle a_{0}, a_{1}, \ldots, a_{n-1}\rangle \in U$, define the hash function as $h_{a}(x) = (\sum_{j=0}^{n-1}a_{j}x_{j}) \mod p$. Let $\mathcal{H} = \{h_{a}\}$. Prove that $\mathcal{H}$ is universal, but not 2-universal.

    \textbf{Solution} First, let's show that $\mathcal{H}$ is universal. Consider two distinct keys $k$ and $l$ from $\mathbb{Z}_{p}$, so that $k \neq l$. For any given hash function $h_{a}$ we let

    \begin{align*}
      r &= \sum_{i = 0}^{n-1}(a_{i}k_{i}) \mod p \\
      s &= \sum_{i = 0}^{n-1}(a_{i}l_{i}) \mod p \\
      r &- s \equiv a(\sum_{i = 0}^{n-1} k_{i} l_{i}) \mod p
    \end{align*}

    Note that $r \neq s$ because $p$ is prime and both $a$ and $(k - l)$ are nonzero modulo $p$, and so their product must also be nonzero modulo $p$. Therefore, when computing any $h_{a} \in \mathcal{H}$, distinct inputs $k$ and $l$ maps to distinct values $r$ and $s$ modulo $p$; there are no collisions yet.

    The probability that distinct keys $k$ and $l$ collide is equal to the probability that $r \equiv s (mod \ m)$ when $r$ and $s$ are randomly chosen as distinct values modulo $p$. For a given value of $r$, of the $p - 1$ possible remaining values for $s$, the number of values $s$ such that $s \neq r$ and $s \equiv r (mod \ m)$ is at most

    \begin{align*}
      \lceil p/m \rceil - 1 &\leq ((p + m - 1)/m) - 1 \\
                            &= (p - 1)/m
    \end{align*}

    The probability that $s$ collides with $r$ when reduced to modulo $m$ is at most $((p - 1)/m)/(p-1) = 1/m$.
    Therefore, for any pair of distinct  values $k, l \in \mathbb{Z}_{p}$,

    \begin{equation*}
      Pr(h_{a}(k) = h_{a}(l)) \leq 1/m
    \end{equation*}

    so $\mathcal{H}$ is universal.

    Secondly, we prove that $\mathcal{H}$ is not 2-universal by counterexample. If we fix $k$ to be $\langle 0, 0, \ldots, 0 \rangle$ and an arbitrary $y$. For any $a \in U$, $\langle h_{a}(x), h_a(y) \rangle = \langle 0, h_{a}(y)\rangle$. Therefore, not all sequence are equally likely to appear and so $\mathcal{H}$ is not 2-universal.

  %%%%%%%%%%%%%
  \item Suppose that we modify $\mathcal{H}$ slightly from part (2): for any $a \in U$ and $b \in \mathbb{Z}_{p}, h'_{ab}(x) = (\sum_{j=0}^{n-1} a_{j}x_{j} + b) \mod p$ and $\mathcal{H}' = \{h'_{ab}\}$, argue that $\mathcal{H}'$ is 2-universal.

    \textbf{Solution} The proof is similar to the previous one but now we want to prove that
    all pairs $\angles{h'_{a,b}(k), h'_{a,b}(l)}$ are equally likely when $h$ is chosen at random.

    Since $k \neq l$, at least one bit differs. To simplify the prove, let's assume that $k_{0} \neq l_{0}$. Let $K = \sum_{i=1}^{n-1} a_{i}k_{i}$ and $L = \sum_{i=1}^{n-1} a_{i}l_{i}$ be the rest of the bits from $k$ and $l$, respectively. Let $r = h'_{a,b}(k)$ and $s = h'_{a,b}(l)$. Define,

    \begin{align*}
      r &\equiv (a_{0}k_{0} + b + K) \mod p\\
      s &\equiv (a_{0}l_{0} + b + L) \mod p\\
      r - s &\equiv a_{0}(k_{0} - l_{0}) + K - L \mod p\\
    \end{align*}

    We want to generate all possible pairs of $\angles{r, s}$ by adjusting $r - s$. Since $k_{0} \neq l_{0}$, $K = L$ and $p$ is prime, the result $r - s$ is completely determined by the choice of $a_{0}$ and $b$. We can determine the right $a_{0}$ by solving the following modular equation

    \begin{align*}
      a_{0} &\equiv (r - s)(k_{0} - l_{o})^{-1} \mod p\\
    \end{align*}

    Notice, $(k_{0} - l_{o})^{-1}$ exists only when $(k_{0} - l_{o})$ is \textit{coprime} with respect to $p$. Since $(k_{0} - l_{o}) \neq 0$ and $p$ is prime, the system has a solution. Once $a_{0}$ is computed, $b$ can be computed by substitution.

    Now, we need to prove that the number of possible pairs $\angles{r, s}$ is $p^{2}$ and each pair is equally likely. Since, there are $p^{2}$ possible choices for $\angles{a_{0}, b}$ and each choice results in a different pair $r - s$, the total number of pairs $\angles{r, s}$ is $p^{2}$. Each $h'_{a,b}$ is chosen uniformly at random from $\mathcal{H}$ and so is $a_{0}$ and $b$. All possible $p^{2}$ pairs are equally likely and by definition $\mathcal{H}$ is 2-universal.

  %%%%%%%%%%%%%
  \item

    \textbf{Solution} Once the adversary intercepts the message, the adversary could use all computational power to compute all possible values for $t$ by computing $\{h(m) | h \in \mathcal{H}\}$. However, since $\mathcal{H}$ is 2-universal, each possible value $t \in \{0, 1, \ldots, p - 1\}$ will appear with probability $1/p$.

    So, the adversary can only \textit{guess} the $h \in \mathcal{H}$ used by Bob and Alice.
    Once the adversary has picked $m'$, the adversary can only guess the value of $t'$. Since all pairs are equally likely, there are $p$ possible values for $t' = h(m')$ and the probability of fooling Bob is $1/p$.
  %%%%%%%%%%%%%
\end{enumerate}

%%%%%%%%%%%%%%%% BIBLIOGRAPHY %%%%%%%%%%%%%%%%%%%%%

% \bibliographystyle{unsrt} % abbrv, aplha, plain, abstract, apa, unsrt,
% \bibliography{refs}

\end{document}
